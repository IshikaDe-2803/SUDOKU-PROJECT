\documentclass{beamer}
\title{The Sudoku Project}
\subtitle{WEEK 2: 1/08/2021 to 8/08/2021}
\author[Ishika | Yashvi]{Ishika De and Yashvi Donga}
\date{August 2021}

\usetheme{Madrid}

\begin{document}
\begin{frame}
     \titlepage
\end{frame}
\begin{frame}
     \frametitle{Agenda}
     \begin{itemize}
          \item Brief Overview
          \item Current Status
          \item Toolchain
          \item Difficulties
          \item Learnings
     \end{itemize}
\end{frame}

\begin{frame}
     \frametitle{Brief Overview}
     The goal of this project is to investigate a variety of algorithms (backtracking, brute force, stochastic search, depth first search) that are capable of solving
standard Sudoku puzzles, of ranging difficulties, in order to learn more about Sudoku
solving techniques.\newline

     We also wanted to create the sudoku solver using OpenCV that will read a puzzle from an image and solve it. We plan on using OpenCVfor multiple programming languages.
\end{frame}

\begin{frame}
     \frametitle{Current Status}   
     \begin{itemize}
          \item Researched about different types of algorithm that can be used and how we can proceed with the project.
          \item Used the backtracking algorithm to solve any type of sudoku grid in Python.
          \item Generated solved sudokus using backtracking algorithm in Python.
		  \item Reasearched about OpenCV library of multiple languages like Java, C++ and Haskell.
 		  \item Used Backtracking Algorithm to solve Sudoku in Java and C++. 
		  \item Genrated Solved Sudoku using backtracking algorithm in Java and C++.
	 \end{itemize}
\end{frame}


\begin{frame}
     \frametitle{Toolchain}
     \begin{itemize}
          \item Languages: Python, Haskell, Elixir, C++, Java.
          \item Open CV - Possible in Python, C++ anf Java.
     \end{itemize}
\end{frame}

\begin{frame}
     \frametitle{Difficulties}
     \begin{itemize}
          \item The backtracking algorithm took some time to implement because we had few challenges implementing the recursive function.
		  \item Removing Spaces from the Solved Sudoku for Generating Sudoku was a challenge. 
\end{itemize}
\end{frame}

\begin{frame}
     \frametitle{Learnings}
     \begin{itemize}
     \item learning about the backtracking Algorithm -  Generating and Solving a Sudoku in Python.
     \item Collaboration and understanding git commands, discovered about VSCode live share.
	 \item Backtracking Algorithm in Java and C++.
	 \item Explaining our code, thought process and ideas to each other.
\end{itemize}         
\end{frame}

\end{document}

